\documentclass[11pt, a4paper]{article}
\usepackage{../mysty}
\renewcommand{\lessontitle}{First-order logic, part~II}
\renewcommand{\fulltitle}{Homework 3}


%%%%%%%%%%%%%%%%%%%%%%%%%%%%%%%%%%%%%%%%%%%%%%%%%%%%%%%%%%%%%%%%%%%%%%%%%%%%%%%%%%%%%%
%%%%%%% START DOCUMENT %%%%%%%%%%%%%%%%%%%%%%%%%%%%%%%%%%%%%%%%%%%%%%%%%%%%%%%%%%%%%%%
%%%%%%%%%%%%%%%%%%%%%%%%%%%%%%%%%%%%%%%%%%%%%%%%%%%%%%%%%%%%%%%%%%%%%%%%%%%%%%%%%%%%%%


\begin{document}
\date{}




\begin{center}
{\Large {\bf \fulltitle}}
\end{center}
\hfill{\bf B04902012 Han Sheng, Liu}
\vspace{0.7cm}


\paragraph*{Question 1.}
Construct a structure $\cA$ and a formula $\varphi$ such that
$\cA \models \forall x \exists y \ \varphi$,
but $\cA \not\models \exists x \forall y \ \varphi$.
\begin{framed}
\solution
Consider $A=\{\T, \F\}$, $\varphi=x\approx y$. \\
For every $x$, there must exist $y$ such that $x \approx$
\end{framed}



\paragraph*{Question 2.}
There are many ways to ``encode''
propositional calculus in first-order logic.
One way is to view a propositional variable $p$ as
a relation symbol of arity $0$
as explained in the class.

Here is another way.
Let $L = \{c\}$ be a vocabulary that consists of only one
constant symbol $c$.
Given a propositional formula $\alpha$ 
using variables $p_1,\ldots,p_k$,
consider the FO formula $\Phi$
obtained from $\alpha$ by 
changing every $p_i$ with an atomic $x_i \approx c$.
For example, if $\alpha$ is 
$$p_1\wedge \neg p_2,$$
then $\Phi$ is 
$$
(x_1 \approx c) \ \wedge \ \neg (x_2 \approx c).
$$
Prove that
$\alpha$ is satisfiable (in the sense of propositional calculus)
if and only if $\Phi$ is.


\paragraph*{Question 3.}
Consider a sentence of the following form:
\begin{eqnarray*}
\Phi & := & \exists x_1 \cdots \exists x_n \ \forall y_1 \cdots \forall y_m \ \varphi,
\end{eqnarray*}
where $\varphi$ is quantifier free and does {\em not} contain any function and constant symbol, and $n,m \geq 1$.

Prove that if $\Phi$ is satisfiable, then there is a structure $\cA$ that satisfies $\Phi$ with $|A|\leq n$. 


\paragraph*{Question 4.}
Let $h:\cA \to \cB$ be a homomorphism.
Define a relation $\sim_h$ on $A$ as follows:
$a \sim_h a'$ if and only if $h(a)=h(a')$.
Prove that $\sim_h$ is an equivalence relation on $A$,
and that $\sim_h$ is, in fact, a congruence in $\cA$.

\paragraph*{Question 5.}
Let $h:\cA\to\cB$ be a strong and surjective homomorphism,
and let $\cA/\!\!\sim_h$ be the factor of $\cA$ modulo $\sim_h$.
Define a function $\xi:A/\!\!\sim_h \to B$,
where $\xi([a]_{\sim_h}) = h(a)$.
Prove that $\xi$ is an isomorphism.




\end{document}

%%%%%%%%%%%%%%%%%%%%%%%%%%%%%%%%%%%%%%%%%%%%%
%%%%%%% END OF DOCUMENT %%%%%%%%%%%%%%%%%%%%%
%%%%%%%%%%%%%%%%%%%%%%%%%%%%%%%%%%%%%%%%%%%%%

