%!TEX program = xelatex
\documentclass[12pt,a4paper]{report}

\usepackage{amssymb}
\usepackage{amsmath}
\usepackage{amsthm}
\usepackage{mathtools}
\usepackage{clrscode3e}
\usepackage{graphicx}
\usepackage{listings}
\usepackage{subfig}
\usepackage{listing}
\usepackage{enumitem}
\usepackage{hyperref}
\usepackage{url}
\usepackage{tcolorbox}
\usepackage{tikz}
\usepackage{tabularx}
\usepackage{array}
\newcolumntype{C}{>$c<$}
\lstset{basicstyle=\ttfamily}
\usetikzlibrary{calc,shapes.multipart,chains,arrows}

\newcommand{\points}[1]{ ($#1$ \textit{points}) } 

\usepackage{fontspec}
%\setmainfont{Gill Sans MT}
%\setmainfont{Helvetica}

\pagestyle{plain}

\linespread{1.3}
\textwidth=17cm \textheight=24cm \voffset=-2cm \hoffset=-1.7cm

\theoremstyle{definition}
\newtheorem{problem}{\textbf{Problem}}
\newtheorem{example}{Example}


\theoremstyle{definition}
\newtheorem{definition}{Definition}

\makeatother
\newcommand{\fontitem}{\Large}
\newcommand{\fontitemi}{\normalsize}
\newcommand{\fontitemii}{\normalsize}

\DeclarePairedDelimiter\ceil{\lceil}{\rceil}
\DeclarePairedDelimiter\floor{\lfloor}{\rfloor}

\def\headline#1{\hbox to \hsize{\hrulefill\quad\lower.3em\hbox{#1}\quad\hrulefill}}
\def\headline#1{\hbox to \hsize{\hrulefill\quad\lower.3em\hbox{#1}\quad\hrulefill}}

\begin{document}

\begin{center}
\textbf{\Large Introduction to Mathematical Logic, Spring 2018\\}
\textbf{\Large Homework 1\\} 
\vspace{5pt}
\textbf{B04902012 Han-Sheng Liu, CSIE, NTU}\\
E-mail: \href{mailto:b04902012@ntu.edu.tw}{\texttt{b04902012@ntu.edu.tw}}\\

\end{center}
\vspace{10pt}
\begin{enumerate}[label=(\arabic*)]
\item
    \begin{enumerate}[label=(\alph*)]
        \item
            \[
            \begin{array}{C|C|C|C}
            $p$ & $q$ & $q\to p$ & $p\to(q\to p)$\\
            \hline
            T & T & T & T\\
            T & F & T & T\\
            F & T & F & T\\
            F & F & T & T
            \end{array}
            \]
        \item
            \[
            \begin{array}{C|C|C|C|C|C}
            $p$ & $q$ & $r$ & $p\to(q\to r)$ & $(p\to q)\to(p\to r)$ & $(p\to(q\to r))\to((p\to q)\to(p\to r))$\\
            \hline
            T & T & T & T & T & T\\
            T & T & F & F & F & T\\
            T & F & T & T & T & T\\
            T & F & F & T & T & T\\
            F & T & T & T & T & T\\
            F & T & F & T & T & T\\
            F & F & T & T & T & T\\
            F & F & F & T & T & T\\
            \end{array}
            \]
        \item
            \[
            \begin{array}{C|C|C|C}
            $p$ & $q$ & $p\lor q$ & $p\to(p\lor q)$\\
            \hline
            T & T & T & T\\
            T & F & T & T\\
            F & T & T & T\\
            F & F & F & T
            \end{array}
            \]
        \item
            \[
            \begin{array}{C|C|C|C}
            $p$ & $q$ & $\neg p\to q$ & $p\to(\neg p\to q)$\\
            \hline
            T & T & T & T\\
            T & F & T & T\\
            F & T & T & T\\
            F & F & F & T
            \end{array}
            \]
    \end{enumerate}
    All formulas are tautology.
\item
    The formulas are constructed as following:
    $\alpha_k = p_{k,1}\lor p_{k,2}\lor p_{k,3}\lor p_{k,4}
                \bigwedge\limits_{1\leq i<j\leq 4}\neg(p_{k,i}\land p_{k,j}
                \bigwedge\limits_{1\leq i\leq 4}
                        )

            

            
    \end{enumerate}
\end{enumerate}

\end{document}
