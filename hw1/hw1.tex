%!TEX program = xelatex
\documentclass[12pt,a4paper]{report}

\usepackage{amssymb}
\usepackage{amsmath}
\usepackage{amsthm}
\usepackage{mathtools}
\usepackage{clrscode3e}
\usepackage{graphicx}
\usepackage{listings}
\usepackage{subfig}
\usepackage{listing}
\usepackage{enumitem}
\usepackage{hyperref}
\usepackage{url}
\usepackage{tcolorbox}
\usepackage{tikz}
\usepackage{tabularx}
\usepackage{array}
\newcolumntype{C}{>$c<$}
\lstset{basicstyle=\ttfamily}
\usetikzlibrary{calc,shapes.multipart,chains,arrows}

\newcommand{\points}[1]{ ($#1$ \textit{points}) } 

\usepackage{fontspec}
%\setmainfont{Gill Sans MT}
%\setmainfont{Helvetica}

\pagestyle{plain}

\linespread{1.3}
\textwidth=17cm \textheight=24cm \voffset=-2cm \hoffset=-1.7cm

\theoremstyle{definition}
\newtheorem{problem}{\textbf{Problem}}
\newtheorem{example}{Example}


\theoremstyle{definition}
\newtheorem{definition}{Definition}

\makeatother
\newcommand{\fontitem}{\Large}
\newcommand{\fontitemi}{\normalsize}
\newcommand{\fontitemii}{\normalsize}

\DeclarePairedDelimiter\ceil{\lceil}{\rceil}
\DeclarePairedDelimiter\floor{\lfloor}{\rfloor}

\def\headline#1{\hbox to \hsize{\hrulefill\quad\lower.3em\hbox{#1}\quad\hrulefill}}
\def\headline#1{\hbox to \hsize{\hrulefill\quad\lower.3em\hbox{#1}\quad\hrulefill}}

\begin{document}

\begin{center}
\textbf{\Large Introduction to Mathematical Logic, Spring 2018\\}
\textbf{\Large Homework 1\\} 
\vspace{5pt}
\textbf{B04902012 Han-Sheng Liu, CSIE, NTU}\\
E-mail: \href{mailto:b04902012@ntu.edu.tw}{\texttt{b04902012@ntu.edu.tw}}\\

\end{center}
\vspace{10pt}
\begin{enumerate}[label=(\arabic*)]
\item
    \begin{enumerate}[label=(\alph*)]
        \item
            \[
            \begin{array}{C|C|C|C}
            $p$ & $q$ & $q\to p$ & $p\to(q\to p)$\\
            \hline
            T & T & T & T\\
            T & F & T & T\\
            F & T & F & T\\
            F & F & T & T
            \end{array}
            \]
        \item
            \[
            \begin{array}{C|C|C|C|C|C}
            $p$ & $q$ & $r$ & $p\to(q\to r)$ & $(p\to q)\to(p\to r)$ & $(p\to(q\to r))\to((p\to q)\to(p\to r))$\\
            \hline
            T & T & T & T & T & T\\
            T & T & F & F & F & T\\
            T & F & T & T & T & T\\
            T & F & F & T & T & T\\
            F & T & T & T & T & T\\
            F & T & F & T & T & T\\
            F & F & T & T & T & T\\
            F & F & F & T & T & T\\
            \end{array}
            \]
        \item
            \[
            \begin{array}{C|C|C|C}
            $p$ & $q$ & $p\lor q$ & $p\to(p\lor q)$\\
            \hline
            T & T & T & T\\
            T & F & T & T\\
            F & T & T & T\\
            F & F & F & T
            \end{array}
            \]
        \item
            \[
            \begin{array}{C|C|C|C}
            $p$ & $q$ & $\neg p\to q$ & $p\to(\neg p\to q)$\\
            \hline
            T & T & T & T\\
            T & F & T & T\\
            F & T & T & T\\
            F & F & F & T
            \end{array}
            \]
    \end{enumerate}
    All formulas are tautology.
\item
    The formulas are constructed as following:
    \begin{align*}
    \alpha_k = &\bigvee\limits_{1\leq i\leq n-1}p_{k,i}\\
               &\bigwedge\limits_{1\leq i<j\leq n-1}\neg(p_{k,i}\land p_{k,j})\\
               &\bigwedge\limits_{1\leq l\leq n}\bigwedge\limits_{1\leq i\leq n-1}\neg(p_{k,i}\land p_{l,i})\\
            X=&\{\alpha_k|1\leq k\leq n\}
    \end{align*}
    $X$ corresponds to $(n-1)$-colorability of an $n$-clique $G$. The construction is a generalized version of the formula set in the proof of \textit{Lemma 3.6}. \\
    $X$ itself is trivially unsatisfiable, since an $n$-clique is not $(n-1)$-colorable. However, every proper subset of $X$ is satisfiable. Intuitively, a proper subset corresponds to a proper induced subgraph of $G$, which is an $m$-clique with $m<n$ and is thus $(n-1)$-colorable.\\
    Here gives an truth assignment for a proper set $Y$ of $X$. Let $|Y|=m$, $m<n$. W.L.O.G, assume that $Y=\{\alpha_k|1\leq k\leq m\}$. A satisfying assignment is
        \begin{equation*}
            p_{k,i}=
                \begin{cases}
                \text{T} &\text{, if } k\leq m \text{ and } k=i.\\
                \text{F} &\text{, otherwise.}\\
                \end{cases}
        \end{equation*}
    That is, color the vertex $i$ inside the subgraph with color $i$, and color the vertex outside the subgraph with nothing.

\item
    \begin{align*}
        X\models\alpha&\Rightarrow X\cup \{\neg\alpha\}\text{ is not satisfiable}\\
                      &\Rightarrow X\cup \{\neg\alpha\}\text{ is not finitely satisfiable}\\
                      &\Rightarrow \exists\text X'\subseteq X\cup\{\neg\alpha\}\text{ s.t. }X'\text{ is finite and non-satisfiable}\\
                      &\Rightarrow X'\cup\{\neg\alpha\}\text{ is a finite subset of $X$ and non-satisfiable}\\
                      &\Rightarrow X'\setminus\{\neg\alpha\}\models\alpha
    \end{align*}
    i.e. the set $X'\setminus\{\neg\alpha\}$ is the desired $X_0$.

\end{enumerate}

\end{document}
